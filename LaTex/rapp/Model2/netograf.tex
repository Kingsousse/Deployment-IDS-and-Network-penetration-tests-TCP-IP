\def\bibname{R�f�rences Internet}
\begin{thebibliography}{MMM90}

\bibitem[1]{1}
Comment �a Marche. \textit{Introduction � la s�curit� informatique}. Disponible sur
\href{http://www.commentcamarche.net/contents/secu/securite-mise-en-oeuvre}{http://www.commentcamarche.net/contents/secu/securite-mise-en-oeuvre} (consult� le 02 Mai 2010)

\bibitem[2]{2}
L'encyclop�die Wikip�dia. \textit{Vuln�rabilit� (informatique)}. Disponible sur
\href{http://wapedia.mobi/fr/Vuln�rabilit�}{http://wapedia.mobi/fr/Vuln�rabilit�} (consult� le 02 Mai 2010)

\bibitem[3]{3}
Comment �a Marche. \textit{ Les scanners de vuln�rabilit�s - Balayage de ports}. Disponible sur
\href{http://www.commentcamarche.net/contents/attaques/sniffers}{http://www.commentcamarche.net/contents/attaques/sniffers} (consult� le 03 Mai 2010)

\bibitem[4]{4}
Sectools. \textit{ Top 11 Packet Sniffers}. Disponible sur
\href{http://www.sectools.org/}{http://www.sectools.org/} (consult� le 06 Mai 2010)

\bibitem[5]{5}
Comment �a Marche.\textit{Audits de vuln�rabilit�}. Disponible sur
\href{http://www.commentcamarche.net/contents/secu/audit-vuln�rabilit�}{http://www.commentcamarche.net/contents/secu/audit-vuln�rabilit�} (consult� le 18 Mai 2010)

\bibitem[6]{6}
L'encyclop�die Wikip�dia. \textit{BackTrack}. Disponible sur
\href{http://fr.wikipedia.org/wiki/BackTrack}{http://fr.wikipedia.org/wiki/BackTrack} (consult� le 19 Mai 2010)

\end{thebibliography}